\documentclass[a4paper,12pt]{article}
\usepackage{amsfonts}
\usepackage{amsmath}
\begin{document}



\section{Problem}

"Imagine taking a number and moving its last digit to the front. 
For example, 1,234 would become 4,123. What is the smallest positive 
integer such that when you do this, the result is exactly double the 
original number? (For bonus points, solve this one without a 
computer.)"

\section{Solution}

Let's begin by writing out numbers in the following form\bigskip

$10^Nd_N+10^{N-1}d_{N-1}+\cdots+10^1d_1+10^0d_0=\sum^N_{n=0}10^nd_n 
\newline \indent \mbox{ where }d_n \in \{0,1,2,\ldots,9\}$\bigskip

\noindent We are looking for a number that satisfies the following 
equality\bigskip

$2\sum^N_{n=0}10^nd_n=10^Nd_0+\sum^N_{n=1}10^{n-1}d_n$\bigskip

\noindent Rearranging we get\bigskip

$\sum^N_{n=1}(2*10^n-10^{n-1})d_n=(10^N-2*10^0)d_0$\bigskip

\noindent Note that $2*10^n-10^{n-1}=19*10^{n-1}$.  As such we rewrite 
the equation as follows\bigskip

$\sum^N_{n=1}10^{n-1}d_n=\frac{(10^N-2*10^0)d_0}{19}$\bigskip

\noindent Since we know that the left side of the equation is an 
integer, we also know that the numerator of the right side had to be 
divisible by 19! As such, we know that the following must be true
\bigskip

$10^N-2\equiv0 \ (mod \ 19)$\bigskip

\noindent [More explination of how to solve this]\bigskip

\noindent ... we get\bigskip

$N=17n+18 \mbox{ where } n \in \mathbb{Z}_{\geq 0}$\bigskip

\noindent This shows us that the smallest possible number that 
satisfies the problem is 18 digits!  Once we set $N=18$ we observe 
that the right side of equation - can only take on 10 different values 
(i.e. each of the 10 possible values for $d_0$). Starting from $0$ we 
find that $d_0=2$ is the first value to satisfy equation -.  We can now
 construct our number\bigskip

$10*\frac{(10^{18}-2)*2}{19}+2=\ldots$

\end{document}
