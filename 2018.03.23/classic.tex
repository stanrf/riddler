\documentclass[a4paper,12pt]{article}
\usepackage{amsfonts}
\usepackage{amsmath}
\begin{document}



\section{Problem}

"Imagine taking a number and moving its last digit to the front. 
For example, 1,234 would become 4,123. What is the smallest positive 
integer such that when you do this, the result is exactly double the 
original number? (For bonus points, solve this one without a 
computer.)"

\section{Solution}

Let's begin by writing out numbers in the following form\bigskip

$10^Nd_N+10^{N-1}d_{N-1}+\cdots+10^1d_1+10^0d_0=\sum^N_{n=0}10^nd_n 
\newline \indent \mbox{ where }d_n \in \{0,1,2,\ldots,9\}$\bigskip

\noindent We are looking for a number that satisfies the following 
equality\bigskip

$2\sum^N_{n=0}10^nd_n=10^Nd_0+\sum^N_{n=1}10^{n-1}d_n$\bigskip

\noindent Rearranging we get\bigskip

$\sum^N_{n=1}(2*10^n-10^{n-1})d_n=(10^N-2*10^0)d_0$\bigskip

\noindent Note that $2*10^n-10^{n-1}=19*10^{n-1}$.  As such we rewrite 
the equation as follows\bigskip

$\sum^N_{n=1}10^{n-1}d_n=\frac{(10^N-2*10^0)d_0}{19}$\bigskip

\noindent Since we know that the left side of the equation is an 
integer, we also know that the numerator of the right side had to be 
divisible by 19. As such, we know that one of the following must 
be true:
\bigskip

$(10^N-2)(mod \ 19) = 0$\bigskip

$or$\bigskip

$d_0(mod \ 19) = 0$\bigskip

\noindent Because we previously defined $d_n \in \{0,1,2,\ldots,9\}$\bigskip,
we know that $d_0(mod \ 19) = 0$ can never be true. Therefore $(10^N-2)(mod \ 19) = 0$ 
must hold true. We can now solve this equation for $N$.\bigskip

\noindent [More explination of how to solve this]\bigskip

\noindent ... we get\bigskip

$N=18n+17 \mbox{ where } n \in \mathbb{Z}_{\geq 0}$\bigskip

\noindent The smallest value that satisfies this equation is $n = 0$. This yields
a of 17 for $N$. Because $N$ respresents the largest power of 10 found in our solution and our
indexing begins at 0, we can deduce that the smallest possible number that 
satisfies the problem is 18 digits. Once we set $N=17$ we observe 
that the right side of equation becomes: \bigskip

$\sum^N_{n=1}10^{n-1}d_n=\frac{(10^17-2*10^0)d_0}{19}

We now see that the left side of the equation can only take on 10 different values 
(i.e. each of the 10 possible values for $d_0$) and represents the first 17 digits
of the solution. Starting from $0$, we find that $d_0=2$ is the first value 
to satisfy the equation.\bigskip

$\frac{(10^17-2*10^0)*2}{19} = 10,526,315,789,473,684$\bigskip

\noindent Tacking on $d_0$ to the end of this number yields the following number
and the solution to the riddle.

$105,263,157,894,736,842$\bigskip

\noindent Move the 2 to the front of the number.\bigskip

$210,526,315,789,473,684$\bigskip

\noindent and 

$210,526,315,789,473,684 = 2*105,263,157,894,736,842$

\end{document}
